\documentclass[a4paper,12pt]{report}
\special{papersize=210mm,297mm}
\usepackage{latexsym,graphicx,epsfig}
\usepackage{cite,subfig}
\usepackage{amsfonts}
\usepackage{amsmath}
\usepackage{amsthm}
\usepackage{amssymb}
\usepackage{amscd}
\usepackage[english]{babel}
\usepackage{simplemargins}
%\usepackage{fancyhdr}
%\setlength{\topmargin}{-10mm} \setlength{\oddsidemargin}{-1mm}
%\setlength{\evensidemargin}{-1mm} \setlength{\textwidth}{6.5in}
%\setlength{\textheight}{9.3in}
%---------------------------------------------------------------

\setlength{\unitlength}{0.9cm}
\newcommand{\ra}   { \rightarrow }
\newcommand{\lra}   { \longrightarrow }
\newcommand{\imsizeA} {18}  %16 cm
\newcommand{\imsizeAh} {9}  %16 cm
\newcommand{\imsizeB} {12}  %8 cm
\newcommand{\imsizeBh} {6}  %8 cm
\newcommand{\imsizeC} {5}  %6 cm
\newcommand{\imsizeD} {9.5}  %4 cm
\newcommand{\ang} {\measuredangle}
\newcommand{\mt}{\mathbf}   % German
\newcommand{\mf}{\mathfrak} % Gothic
\newcommand{\mc}{\mathcal}  % special
\newcommand{\mb}{\mathbb}   % same as special
\newcommand{\Lchusk} {\left(\begin{array}{c}L\\ k \end{array}
	\right)}
\newcommand\vect[1]{{\bf#1}}
\newcommand\matr[1]{{\bf#1}}

\numberwithin{equation}{section}
%\newenvironment{proof}{\noindent{\bf Proof:}}{\\ \hspace*{\fill}$\Box$ \par}

\newcommand{\w  }   { ^{-1} }
\newcommand{\overC} { _{\mt{C}} }
\newcommand{\tz}    { \otimes }
\newcommand{\ds}    { \displaystyle}
\def \mg     {{g^{-m}}}
\def \ih     {{\frac{1}{|H|^2}}}
\def \Mmi    {{M^{m-1}        }}
\newtheorem{corollary}{Corollary}
\newtheorem{proposition}{Proposition}
\newtheorem{remark}{Remark}
\newtheorem{theorem}{Theorem}
\newtheorem{assumption}{Assumption}

%------------------------------------------------------------------------------------------
%    \setleftmargin{0.25in}
%    \setrightmargin{0.55in}
%    \settopmargin{0.73in}
%    \setbottommargin{0.48in}
%\linespread{1.1} \selectfont

%------------------------------------------------------------------------------------------

\setleftmargin{1.0in}
\setrightmargin{1.2in}
\settopmargin{1.0in}
\setbottommargin{0.6in}
\linespread{1.5} \selectfont
%------------------------------------------------------------------------------------------

\begin{document}
	\title{Solutions for Baez and Minuanin, Gauge Fields, Knots and Gravity}
	\author{J. Renes, J. Shtok}
	
	\maketitle

	
\subsection{Chapter 6}

\begin{p}%80
{Show that $\omega = (x dy-ydx)/(x^2+y^2)$ is closed. Show that $\int_{\gamma_0}\omega=-\pi$ while $\int_{\gamma_1}\omega=\pi$. }
\end{p}

{Let $r=x^2+y^2$. Then $\partial_i r=2x_i$. Now $d\omega=\partial_x(x/r) dy\wedge dx-\partial_y(y/r)dx\wedge dy=(1/r-2x^2/r^2)dy\wedge dx+(1/r-2y^2/r^2)dy\wedge dx
=(y^2-x^2)/r^2dy\wedge dx+(x^2-y^2)/r^2 dy\wedge dx=0.$ Turning to the integrals,
$\gamma_0$ is the upper half of the unit circle, traversed from (-1,0) to (1,0). We can
parameterize this path as $\gamma_0(t)=(\cos \pi(1{-}t),\sin \pi(1{-}t))$, and thus
$\gamma'_0(t)=\pi(\sin\pi(1{-}t),-\cos\pi(1{-}t))$. In these coordinates and on this path $\omega=(-y/r,x/r)=(-\sin \pi(1{-}t),\cos \pi(1{-}t))$, so $\int_{\gamma_0}\omega=\int_0^1 \omega_{\gamma_0(t)}(\gamma'_0(t)) {\rm d}t=-\pi$.
The other curve is $\gamma_1(t)=(\cos \pi(1{-}t),-\sin \pi(1{-}t))$, so $\gamma'_1(t)=\pi(\sin\pi(1{-}t),\cos\pi(1{-}t))$ and $\omega_{\gamma_1(t)}=(\sin \pi(1{-}t),\cos \pi(1{-}t))$, leading to $\int_{\gamma_1}\omega=\pi$.}

\begin{p}%81
{Show that $\R^n$  is simply connected by exhibiting an explicit formula
for a homotopy between any two paths between arbitrary points $p,q\in\R^n$.}
\end{p}
{Let
$f(t)$ and $g(t)$ be paths from $p$ to $q$. Then define $h(s,t)=sf(t)+(1{-}s)g(t)$. Clearly $h(0,t)=f(t)$ and $h(1,t)=g(t)$ and since all the intervening points are elements of $\R^n$ 
for all $t$, $h$ is a homotopy between $f$ and $g$.}


\begin{p}%82
{Show that a 1-form $E$ is exact iff $\int_\gamma E=0$ for all loops
$\gamma$. (Hint: if $E$ is not exact, show that there are two smooth paths $\gamma,\gamma'$ from some point $x\in M$ to some point $y\in M$ such that $\int_\gamma \om\neq \int_{\gamma'} E$. Use these paths to form a loop, perhaps only piecewise smooth.}
\end{p}

{Start with the `if' statement and let $E=d\phi$ be an exact form and $\gamma$ a loop starting and ending at $p$. Then 
$\oint_\gamma E=\int_0^1 d\phi(\gamma'(t)){\rm d}t=\int_0^1\gamma'(t)[\phi]{\rm d}t=\int_0^1\frac{\rm d}{{\rm d}s}\phi(\gamma(s))_{s=t}{\rm d}t=\int_0^1[\phi(\gamma(t))]'{\rm d}t=\phi(\gamma(1))-\phi(\gamma(0))=\phi(p)-\phi(p)=0.$
Now the `only if' part, which we will show by proving the contrapositive.
If $E$ is closed but not exact, then the manifold must not be simply connected. Then choose two points $x$ and $y$ such that the paths $\gamma,\gamma'$ have no homotopy between them, i.e.~$\gamma$ and $\gamma'$ go on either side of the ``hole'' from $x$ to $y$; together they encircle the hole. Additionally, we must be able to
find some points $x,y$ and paths $\gamma,\gamma'$ such that $\int_\gamma E\neq \int_{\gamma'}E$ or else we could use this integral to define $\phi$ for which $E=d\phi$. 
Then defining $\tilde{\gamma}$ to be the path taken by following $\gamma$ forwards and then $\gamma'$ backwards, we have found a path $\tilde{\gamma}$ such that  $\oint_{\tilde{\gamma}} E\neq 0$.}

\begin{p}%{83}
{For any manifold $M$, show the manifold $S^1\times M$ is not simply connected by finding a 1-form on it that is closed by not exact.}{Picking coordinates $(t,x^j)$ consider the 1-form $\om=$(1,0,\dots,0). Clearly $d\om=0$ so the form is
closed, but it is not exact. There is no continuous function $\phi$ defined on the whole manifold such that $\om=d\phi$, since if we try to define it using $\int_\gamma \om$ where
$\gamma$ is a path around $S^1$ and fixed on $M$, we obtain $\phi(t,x^j)=t$, which is multiple-valued.}
\end{p}

\begin{p}%{84}
{Let the $n$-disk $D^n$ be defined as $D^n=\{(x_1,\dots,x_n):x_1^2+\dots+x_n^2\leq 1\}$. Show that $D^n$ is an $n$-manifold with
boundary in an obvious sort of way.}
\end{p}
{Locally the neighborhood of each point on the unit circle $S^n$ already looks a lot like $H^{n}$ where the half space is defined by the tangent hyperplane. In particular, we can make
suitable charts by first using spherical coordinates to describe the points in the disk and then for each point,
assuming that it lies at the north pole and expanding the coordinates in a Taylor series in $\theta$ to first order. In two dimensions we then obtain $r(\cos \theta,\sin\theta)\rightarrow r(1,\theta)$ and in three dimensions
$r(\cos\theta,\sin\theta\cos\varphi,\sin\theta\sin\varphi)\rightarrow r(1,\theta\cos\varphi,\theta\sin\varphi)$. We then obtain a cylindrical coordinatization of the neighborhood of the selected point.}

\begin{p}%{85}
{Check that the definition of tangent vectors in Chapter 3 really does imply that the tangent space at any point on the boudary of an $n$-dimensional manifold with boundary is an $n$-dimensional vector space.}
\end{p}
{Tangent vectors are simply maps from $C^\infty$ to $\R$ obeying linearity and the Leibniz rule. If we first map the point on the boundary to the boundary of $H^n$, where $x^n\geq 0$ is the special coordinate, 
then obviously 
we still have all the derivatives in the directions $x_1,\dots,x_{n{-}1}$. The derivative in the $x_n$ direction still works as well, since the functions must be smooth to a small region outside the boundary, for which $-\varepsilon<x_n<0$. }

\begin{p}%{86}
{For the mathematically-inclined reader: prove that $\int_M\om$ is independent of the 
choice of charts and partition of unity.}
\end{p}
...

\begin{p}%{87}
{Show that $\partial D^n=S^{n{-}1}$.}
\end{p}
{In exercise 84 we showed a collection of charts in which
at every point on the surface of the sphere nearby surface points are mapped to points on $\partial H^n$, thought of as the tangent hyperplane. Thus, surface points are on the boundary. No other points can be on the boundary since they are ``surrounded'' by other points in the manifold. Consider mapping a point in the interior to a point on $\partial H^n$. Not all open sets containing the point are mapped to open sets of $H^n$, and thus the chart transition function isn't smooth (isn't even continuous).}

\begin{p}%{88}
{Let $M=[0,1]$. Show that Stokes' theorem in this case is equivalent to the fundamental theorem
of calculus $\int_0^1 {\rm d}x\, f'(x)=f(1)-f(0)$.}
\end{p}
{Consider the 1-form $\om=df$. By Stokes' theorem $\int_M \om=\int_0^1 df=\int_{\partial M}f$. The boundary $\partial M$ is manifestly equal to $\{0,1\}$; only the orientation remains to be determined.  $dx$ defines the orientation by defining increasing $x$ to be outward-facing at 1 and inward-facing at 0. Thus $\int_{\partial M}f=f(1)-f(0)$.}

\begin{p}%{89}
{Let $M=[0,\infty)$, which is not compact. Show that without the assumption that $f$ vanishes outside a compact set, Stokes' theorem may not apply. (Hint: in this case Stokes' theorem says $\int_0^\infty {\rm d}x\, f'(x)=-f(0)$.)}
\end{p}
{Choose the function $f'(x)=1$. Then $f(x)=x$. The integral clearly diverges, whereas 
by Stokes' theorem it would be zero. The function $f'(x)=e^{-x}$ works, though.}

\begin{p}%{90}
{Show that any submanifold is a manifold in its own right in a natural way.}
\end{p}
{A manifold is defined by having charts $\varphi_\alpha$ which smoothly map the open sets $U_\alpha$ to $\R^k$, where smooth means that the transition function $\varphi_\alpha\circ\varphi_\beta^{-1}$ is smooth 
(infinitely differentiable) where defined. For a submanifold $S\subset M$ we can define the open sets to 
be of the form $V_\alpha=S\cap U_\alpha$; this is the induced topology. Charts $\varphi_\alpha$ of $M$ are
guaranteed to satisfy $S\cap U_\alpha=\varphi_\alpha^{-1}\R^k$, which means that every open set in $S_\alpha$ is the preimage of some hyperplane $\simeq\R^k$ under the chart $\varphi_\alpha$. So we can
just define $\vartheta_\alpha$ to be this map from $S_\alpha$ to $\R^k$. The transition functions
$\vartheta_\alpha\circ\vartheta_\beta^{-1}$ are smooth because they are equivalent to the transition
functions $\varphi_\alpha\circ\varphi_\beta^{-1}$, restricted to the corresponding input and output 
hyperplanes.}

\begin{p}%{91}
{Show that $S^{n{-}1}$ is a compact submanifold of $\R^n$.}
\end{p}
{Use the maps in exercise 84, fixing $r=1$.}


\begin{p}%{92}
{Show that any open subset of a manifold is a submanifold.}
\end{p}
{Isn't this just exercise 4?}

\begin{p}%{93}
{Show that if $S$ is a $k$-dimensional submanifold 
with boundary of $M$, then $S$ is a manifold with boundary in a natural way. 
Moreover, show that $\partial S$ is a $(k{-}1)$-dimensional 
submanifold of $M$.}
\end{p}
{Same as exercise 90 except that the hyperplanes have boundaries.
That doesn't change the fact that the transition functions will be smooth. $\partial S$
is also a manifold using the charts from $S$ with the modification that the image of
the chart is now just the boundary of the hyperplane used before. Again, the transition
functions will be smooth, since they are just restrictions of smooth maps. }

\begin{p}%{94}
{Show that $D^n$ is a submanifold of $\R^n$ in this sense.}
\end{p}
{Isn't this just 
exercise 84?}

\begin{p}%{95}
{Suppose that $S\subset \R^2$ is a 2-dimensional compact orientable submanifold with boundary. Work out what Stokes' theorem says when aplied to a 1-form on $S$. This is sometimes called
Green's theorem.}
\end{p}
{Let $\om=\alpha dx+\beta dy$. Then $d\om=(\partial_x \alpha-\partial_y\beta)dx\wedge dy$. Thus $\int_{\partial M}\alpha dx+\beta dy=\int_M (\partial_x \alpha-\partial_y\beta)dx\wedge dy$. To evaluate the lefthand side one should
make a coordinate transformation by regarding $\partial M$ as a curve $\gamma(t)$, the
inverse of which we can regard as a map from $\partial M$ to $\R$. This allows us 
to express $dx$ and $dy$ in terms of the coordinate 1-form $dt$: $dx=\frac{d\gamma^1}{dt}dt$ and likewise for $dy$.}

\begin{p}%{96}
{Suppose that $S\subset \R^3$ is a 2-dimensional compact orientable submanifold with boundary. Show that Stokes' theorem aplied to $S$ boils down to the classic Stokes's theorem.}
\end{p}
{Suppose we work in a patch of $S$ with coordinates $x,y,z$ such that $z$ is fixed, so that $z=c$ defines $S$. On this patch let $\om=v_i dx^i$, where $i=1,2$. Then $d\om=\partial_j v_i dx^j\wedge dx^i$ for $i,j=1,2$. This can 
be written $d\om=(\partial_2 v_1-\partial_1 v_2)dx^1\wedge dx^2$; the corresponding integral $\int_S d\om=\int_S (\nabla\times \vec{v})\cdot \hat{z}dA$, where $A$ is the area 
(volume) element of $S$. The boundary of $S$ must be specified by a curve $\gamma(t)=(x^1(t),x^2(t))$. It then follows that $\om=v_i\frac{\partial x^i}{\partial t}dt$ so that the integral can be written $\int_{\partial S}\om=\int \vec{v}\cdot d\vec{\ell}$ with $\vec{\ell}=\frac{\partial x^i}{\partial t}\hat{x}_i$. Thus we have
 $\int_S (\nabla\times \vec{v})\cdot d\vec{A}=\int \vec{v}\cdot d\vec{\ell}$, the usual Stokes' theorem.}

\begin{p}%{97}
{Suppose that $S\subset \R^3$ is a 3-dimensional compact orientable submanifold with boundary. Show that Stokes's theorem aplied to $S$ is equivalent to Gauss's theorem, also known as the divergence theorem.}
\end{p}
{Let $\om=\alpha dx\wedge dy+\beta dz\wedge dx+\gamma dy\wedge dz$. Then $d\om=\partial_z \alpha dz\wedge dx\wedge dy+\partial_y \beta dy\wedge dz\wedge dx+\partial_x \gamma dx\wedge dy\wedge dz=(\partial_z\alpha+\partial_y\beta+\partial_x \gamma)dx\wedge dy\wedge dz$. Now $\int_{\partial M}\om=\int_M d\om$, the righthand side of which is simply $\int_S(\partial_z\alpha+\partial_y\beta+\partial_x \gamma) dxdydz$. If we define
$v\equiv\star\om=\alpha dz+\beta dy+\gamma dx$, then this integral is just $\int \partial_iv^i dxdydz$, i.e.~the integral of the gradient of the vector $\vec{v}$. Meanwhile, 
consider a patch of $\partial S$ covered by coordinates $(s,t)$; that is, a specification of the points on $\partial S$: $x(s,t), y(s,t),$ and $z(s,t)$. Letting $dA=ds\wedge dt$ be the volume element on $\partial S$, then $dx\wedge dy=(\frac{\partial x}{\partial s}\frac{\partial y}{\partial t}-\frac{\partial x}{\partial t}\frac{\partial y}{\partial s})dA$, and similarly for the other 2-forms. Then the lefthand integrand can be written as $\vec{v}\cdot d\vec{A}$ where the vector $\vec{A}=
\epsilon_{ijk}\hat{x}_i(\frac{\partial x_j}{\partial s}\frac{\partial x_k}{\partial t})$.}

\begin{p}%{98}
{Show that the pullback of a closed form is closed and the pullback of an exact form is exact.}
\end{p}
{The pullback and exterior derivative commute, so if $\om$ is closed, $d\om=0$ and $\phi^*\om$ is also closed, since $d\phi^*\om=\phi^* (d\om)=0$. 
Similarly, if $\om$ is exact, then so is $\phi^*\om$: $\phi^*\om=\phi^*(d\alpha)=d(\phi^*\alpha)$.}

\begin{p}%{99}
{Show that given any map $\phi:M\rightarrow M'$ there is a linear map 
from $H^p(M')$ to $H^p(M)$ given by $[\om]\mapsto[\phi^*\om]$, where
$\om$ is any closed $p$-form on $M'$. Call this linear map $\phi^*:H^p(M')\rightarrow H^p(M)$. Show that if $\psi:M'\rightarrow M''$ is another map, then $(\psi\phi)^*=\phi^*\psi^*$.}
\end{p}
{The linear map is just the pullback:
since it preserves closed and exact forms, it preserves equivalence classes of closed forms. If
$\om,\om'$ are closed, so are $\phi^*\om$ and $\phi^*\om'$ and if $\om-\om'=d\mu$, then $\phi^*\om-\phi^*\om'=\phi^*d\mu=d\phi^*\mu$. And the pullback of the composition of maps is the reverse-ordered composition of pullbacks.}

\begin{p}%{100}
{Show that $\star j=f(r)r\,dr\wedge d\theta$ for $j=f(r)dz$ and $d\theta=\om$ as in exercise 80.}
\end{p}
{$\star j=f(r)\star dz=f(r)dx\wedge dy$. Now note that $r dr\wedge d\theta=(xdx+ydy)(xdy-ydx)/r^2=dx\wedge dy$ since $r dr=xdx+ydy$.}

\begin{p}%{101}
{Show that $\star d\theta=\frac{1}{r}dz\wedge dr$.}{$\star d\theta=(x\star dy-y\star dx)/r^2=(xdz\wedge dx-ydy\wedge dz)/r^2=(dz\wedge(xdx+ydy))/r^2=dz\wedge dr/r$.}
\end{p}

\begin{p}%{102}
{Check that $d\star B=\star j$ holds iff $g'(r)=rf(r)$ for $B=\frac{g(r)}{r} dz\wedge dr$.}
\end{p}
{$d\star B=d(g(r)d\theta)=g'(r)dr\wedge d\theta=\star j=rf(r)dr\wedge d\theta.$ Thus $g'(r)=rf(r)$ is necessary and sufficient.}

\begin{p}%{103}
Using the fact that the $n$-dimensional torus $T^n$ is the product of $n$ copies of $S^1$, construct $n$ closed but not exact $1$-forms $d\theta_1$, $d\theta_2$, \dots, $d\theta_n$. Hint: define maps $p_i:T^n\to S^1$ corresponding to projection down to the $i$th coordinate, where $1\leq i\leq n$, and let $d\theta_i=p_i^*d\theta$.
\end{p}
{Following the hint, what remains to show is that the $d\theta_1$ (for example) is a closed, non-exact 1-form on $T^n$. Recall from Exercise 80, the 1-form $d\theta$ on $S^1$ defined in cartesian coordiantes by $d\theta = \dfrac{xdy-ydx}{x^2+y^2}$. Integral over this 1-form  was shown to have different values on two different pathes on $S^1$; for a positive half circle $\gamma_0$, $\,\int_{\gamma_0}d\theta = \pi$, and for the negative half-circle, $\int_{\gamma_1}d\theta = -\pi$.
Now, consider a path $\gamma_0^T$ on the $n-$torus $T^n$ given by travelling the path $\gamma_0$ on the first $S_1-component$ and keeping the others at zero:
$$
\gamma_0^T = \{\gamma_0, 0, ..., 0\}.
$$
From the definition of $d\theta_i=p_i^*d\theta$, $d\theta_1$ is a one-form defined on the projection of the torus on the 1-st $S^1$ component, so 
$$ 
\int_{\gamma_i^T}d\theta_i = \int_{\gamma_i}d\theta, \qquad i=0,1.
$$
}
This transferes the fact $d\theta$ is not exact (different integral values) to the 1-forms on the torus. 

\begin{p}
\end{p}


\begin{p}
\end{p}


\begin{p}
\end{p}


\begin{p}
\end{p}


\begin{p}
\end{p}


\begin{p}
\end{p}


\begin{p}
\end{p}


\begin{p}%111
\end{p}


\end{document}
